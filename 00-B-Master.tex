
\section{Section 3 Logic}
\subsection{Logical Operations}
\begin{itemize}
\item $\neg p$ the negation of proposition $p$.
\item $p \wedge q$ Both propositions p and q are simultaneously true (Logical State AND)
\item $p \vee q $ One of the propositions is true, or both (Logical State : OR)
\item $p \otimes q$ Only one of the propositions is true (Logical State : exclusive OR (i.e XOR)
\end{itemize}
\begin{center}
\begin{tabular}{|c|c|c|c|c|}
\hline
p & q & $p \vee q$ & $q \wedge p$ & $p \otimes q$ \\
\hline
0 & 0 & 0 & 0 & 0 \\
0 & 1 & 1 & 0 & 1\\
1 & 0 & 1 & 0 & 1 \\
1 & 1 & 1 & 1 & 0\\
\hline
\end{tabular}
\end{center}


%---------------------------%

% Propositions
% Logical Operations
% Conditional Connectives
% Logical Proofs
% De Morgan's Law

%-----------------------------------%


\begin{itemize}
	\item Contra-positives
\end{itemize}


% \frametitle{Logical Notation}

\begin{center}
	\begin{tabular}{|c|c|}
		Logical & Negation \\ \hline
		$p$	& $\neg p$ \\ \hline
		$T$	& $F$      \\ \hline
		$F$	& $T$      \\ \hline
	\end{tabular}
\end{center}


%-----------------------------------%

\section{Logic}
\begin{itemize}
	\item The NOT operator $\neg$ (also known as the negation operator)
\end{itemize}
\subsection{Truth Tables}
{
	\begin{center}
		\begin{tabular}{|c|c||c|c||c|c|}
			\hline $p$ & $q$ & $p \wedge q$ & $p\vee q$ & $\neg p$  & $\neg (p \wedge q)$ \\ 
			\hline 0 & 0 & 0 & 0 & 1 & 1 \\ 
			\hline 0 & 1 & 0 & 1 & 1 & 1 \\ 
			\hline 1 & 0 & 0 & 1 & 0 & 1 \\ 
			\hline 1 & 1 & 1 & 1 & 0 & 0 \\ 
			\hline 
		\end{tabular} 
	\end{center}
}
%-------------------------%

\subsection*{Prepositional Logic}

\section{Section 3 Logic}
\subsection{Logical Operations}
\begin{itemize}
\item $\neg p$ the negation of proposition $p$.
\item $p \wedge q$ Both propositions p and q are simultaneously true (Logical State AND)
\item $p \vee q $ One of the propositions is true, or both (Logical State : OR)
\item $p \otimes q$ Only one of the propositions is true (Logical State : exclusive OR (i.e XOR)
\end{itemize}
\begin{center}
\begin{tabular}{|c|c|c|c|c|}
\hline
p & q & $p \vee q$ & $q \wedge p$ & $p \otimes q$ \\
\hline
0 & 0 & 0 & 0 & 0 \\
0 & 1 & 1 & 0 & 1\\
1 & 0 & 1 & 0 & 1 \\
1 & 1 & 1 & 1 & 0\\
\hline
\end{tabular}
\end{center}
%---------------------------------------------------------%
\section{Conditional Connectives}
Construct the truth table for the proposition $p \rightarrow q$.

\begin{center}
\begin{tabular}{|c|c|c|c|}
\hline
p & q & $p \rightarrow q$ & $q \rightarrow p$ \\
\hline
0 & 0 & 1& 1 \\
0 & 1 & 1 & 0 \\
1 & 0 & 0 & 1 \\
1 & 1 & 1 & 1 \\
\hline
\end{tabular}
\end{center}


\section{Section 3 Logic}
\subsection{Logical Operations}
\begin{itemize}
\item $\neg p$ the negation of proposition $p$.
\item $p \wedge q$ Both propositions p and q are simultaneously true (Logical State AND)
\item $p \vee q $ One of the propositions is true, or both (Logical State : OR)
\item $p \otimes q$ Only one of the propositions is true (Logical State : exclusive OR (i.e XOR)
\end{itemize}
\begin{center}
\begin{tabular}{|c|c|c|c|c|}
\hline
p & q & $p \vee q$ & $q \wedge p$ & $p \otimes q$ \\
\hline
0 & 0 & 0 & 0 & 0 \\
0 & 1 & 1 & 0 & 1\\
1 & 0 & 1 & 0 & 1 \\
1 & 1 & 1 & 1 & 0\\
\hline
\end{tabular}
\end{center}
%---------------------------------------------------------%
\section{Conditional Connectives}
Construct the truth table for the proposition $p \rightarrow q$.

\begin{center}
\begin{tabular}{|c|c|c|c|}
\hline
p & q & $p \rightarrow q$ & $q \rightarrow p$ \\
\hline
0 & 0 & 1& 1 \\
0 & 1 & 1 & 0 \\
1 & 0 & 0 & 1 \\
1 & 1 & 1 & 1 \\
\hline
\end{tabular}
\end{center}


%-------------------------------------------------------------------------%
\newpage

x

%--------------------------------------%

\section{Logic Proposition}

Let $p$, $q$ and $r$ be the following propositions concerning integers $n$ (where $n>1$):

\begin{itemize}
\item $p$ : n is a prime factor of 36 %(2)
\item $q$ : n is a prime factor of 4 %(2)
\item $r$ : n is a prime factor of 9 %(3)
\end{itemize}


\begin{center}
\begin{tabular}{|c||c|c|c|}
\hline 
\phantom{spa} \textbf{n} \phantom{spa}	& \phantom{spa}	\textbf{p} \phantom{spa}	& \phantom{spa}	\textbf{q} \phantom{spa}	& \phantom{spa}	\textbf{r} \phantom{spa}	\\ \hline \hline
1	&	1	&	1	&	1	\\ \hline
2	&	1	&	0	&	1	\\ \hline
3	&	0	&	1	&	1	\\ \hline
%4	&	1	&	0	&	1	\\ \hline
%6	&	0	&	0	&	1	\\ \hline
%9	&	0	&	1	&	1	\\ \hline
%12	&	0	&	0	&	1	\\ \hline
%18	&	0	&	0	&	1	\\ \hline
%36	&	0	&	0	&	1	\\
\hline 
\end{tabular} 
\end{center}

%------------------------------------------------------------------------------- %


{Logical Propostions}

For each of the following compound statements, express it using the propositions $p$,$q$ and $r$, and the appropriate logical symbols, then given the truth table for it,

\begin{itemize}
\item[1)] If n is a prime factor of 36, then n is a prime factor of 4 or n is a prime factor of 9
\item[2)] If n is a prime factor of 4 or n is a prime factor of 9, then  n is a prime factor of 36
\end{itemize}





{Proof With Truth Tables}

Let $p$ and $q$ be propositions. Use\textbf{\textit{Truth Tables}} to prove that

\[ p \rightarrow q \equiv \neg q \rightarrow \neg p\]


%-------------------------------------- %
{Proof With Truth Tables}

%% - \vspace{-1cm}
\textbf{Important}\\ 
Remember to make a comment at the end to say why the table proves that the two statements are logically equivalent. \\ For example : \emph{``Since the relevant columns are identical, then it can be said that both sides of the equation are equivalent"}.


%-------------------------------------- %
{Proof With Truth Tables}

%% - \vspace{-1cm}
Left hand side of expression : $p$ \textit{implies} $q$.
\[p \rightarrow q\]
\begin{center}
\begin{tabular}{|c|c||c|}
\hline  \phantom{spa}p\phantom{spa}&  \phantom{spa}q\phantom{spa}& \phantom{sp}$p \rightarrow q$ \phantom{sp} \\ 
\hline  0&  0&  1\\ 
\hline  0&  1&  1\\ 
\hline  1&  0&  0\\ 
\hline  1&  1&  1\\ 
\hline 
\end{tabular} 
\end{center}



{Proof With Truth Tables}

%% - \vspace{-1cm}
Right hand side of expression : \textit{not-q} \textit{implies} \textit{not-p}
\[\neg q \rightarrow \neg p\]
\begin{center}
\begin{tabular}{|c|c||c|c|c|}
\hline  \phantom{sp}p\phantom{sp}&  \phantom{sp}q\phantom{sp}&\phantom{sp} $\neg q$ \phantom{sp} & \phantom{sp} $\neg p \phantom{sp}$ & $\neg q \rightarrow \neg p$ \\ 
\hline  0&  0& 1& 1& 1\\ 
\hline  0&  1& 0& 1& 1\\ 
\hline  1&  0& 1& 0& 0\\ 
\hline  1&  1& 0& 0& 1\\ 
\hline 
\end{tabular}
\end{center}


Side by Side
\[ p \rightarrow q \equiv \neg q \rightarrow \neg p\]
\bigskip
{ 
\hspace{0.5cm} \begin{tabular}{|c|c||c|}
\hline  p&  q& $p \rightarrow q$ \\ 
\hline  0&  0&  1\\ 
\hline  0&  1&  1\\ 
\hline  1&  0&  0\\ 
\hline  1&  1&  1\\ 
\hline 
\end{tabular} \hspace{0.5cm} \begin{tabular}{|c|c||c|c|c|}
\hline  p&  q& $\neg q$ & $\neg p$ & $\neg q \rightarrow \neg p$ \\ 
\hline  0&  0& 1& 1& 1\\ 
\hline  0&  1& 0& 1& 1\\ 
\hline  1&  0& 1& 0& 0\\ 
\hline  1&  1& 0& 0& 1\\ 
\hline 
\end{tabular}
}\\
%% - \vspace{0.5cm}
(only ``difference" is first and last rows)


%--------------------------------------------------- %
%--------------------------------------------------- %
%\subsection*{1.3 Membership Tables for Laws}
%\emph{Page 44 (Volume 1) Q8.
%Also see Section 3.3 Laws of Logic.}\\
%--------------------------------------%



{Laws of Logic}

%% - \vspace{-1cm}
\textbf{Solutions}
\begin{center}

\begin{tabular}{|c|c||c||c|}
\hline  \phantom{sp}p\phantom{sp}&  \phantom{sp}T\phantom{sp}& $p \vee T$ & $ p \wedge T$ \\ \hline
\hline  0 & 1 & 1 & 0 \\ 
\hline  1 &  1 & 1 & 1 \\ 
\hline 
\end{tabular} 
\end{center}
\begin{itemize}
\item[(i)] $p \vee F \equiv T$
\item[(ii)] $p \wedge T \equiv p$
\end{itemize}




{Laws of Logic}

%% - \vspace{-1cm}
Construct a truth table for each of the following compound statement and hence find simpler propositions to which it is equivalent.


\begin{itemize}
\item[(iii)] $p \vee F$
\item[(iv)] $p \wedge F$
\end{itemize}

%--------------------------------------------------- %
%--------------------------------------------------- %

{Laws of Logic}

%% - \vspace{-1cm}
\textbf{Solutions}
\begin{center}

\begin{tabular}{|c|c||c|c|}
\hline  \phantom{sp}p\phantom{sp}&  \phantom{sp}F\phantom{sp}& $p \vee F$ & $ p \wedge F$ \\ \hline
\hline  0 & 0 &  &  \\ 
\hline  1 &  0 &  &  \\ 
\hline 
\end{tabular} 

\end{center}

%--------------------------------------------------- %
%--------------------------------------------------- %

{Laws of Logic}

%% - \vspace{-1cm}
\textbf{Solutions}
\begin{center}
\begin{tabular}{|c|c||c|c|}
\hline  \phantom{sp}p\phantom{sp}&  \phantom{sp}F\phantom{sp}& $p \vee F$ & $ p \wedge F$ \\ \hline
\hline  0 & 0 & 0 & 0 \\ 
\hline  1 &  0 & 1 & 0 \\ 
\hline 
\end{tabular} 

\end{center}
\begin{itemize}
\item[(iii)] $p \vee F = p $
\item[(iv)] $p \wedge F = F $
\end{itemize}

%--------------------------------------------------- %
%--------------------------------------------------- %
%
%{Laws of Logic}
%
%\begin{itemize}
%\item Logical OR:  $p \vee F = p $
%\item Logical AND: $p \wedge F = F $
%\end{itemize}
%



\[\mbox{Discrete Maths :  Logic}\]
\[\mbox{Contra-positives}\]
\bigskip

\[\mbox{www.Stats-Lab.com}\]
\[\mbox{Twitter: @StatsLabDublin}\]



%--------------------------------------------------- %
%--------------------------------------------------- %
%\subsection*{1.4 Propositions}
%\textbf{Page 67 Question 9}

{Contra-positive}

Write the contra-positive of each of the following statements:

\begin{itemize}
\item If n= 12, then n is divisible by 3.
\item If n=5, then n is positive.
\item If the quadrilateral is square, then four sides are equal.
\end{itemize}


{Contra-positives}

\textbf{Solutions}
\begin{itemize}
\item If n is not divisible by 3, then n is not equal to 12.
\item If n is not positive, then n is not equal to 5.
\item If the four sides are not equal, then the quadrilateral is not a square.
\end{itemize}


%--------------------------------------------------- %
%--------------------------------------------------- %


\[\mbox{Discrete Maths :  Logic}\]
\[\mbox{Truth Sets}\]
\bigskip

\[\mbox{www.Stats-Lab.com}\]
\[\mbox{Twitter: @StatsLabDublin}\]



%--------------------------------------------------- %
%--------------------------------------------------- %
%\subsection*{1.5 Truth Sets}

{Truth Sets}


\textbf{2009} 

Let $n = \{1, 2,3,4, 5,6,7, 8, 9\}$ and let $p$ and  $q$ be the following propositions concerning the integer $n$.
\begin{itemize}
\item p: n is even, 
\item q: $n\geq 5$.
\end{itemize}
By drawing up the appropriate truth table find the truth set for each of the
propositions $p \vee \neg q$ and $ \neg q \rightarrow p$


%--------------------------------------------------- %
%--------------------------------------------------- %
%\subsection*{1.5 Truth Sets}

{Truth Sets}

%% - \vspace{-0.5cm}
\begin{center}
\begin{tabular}{|c||c|c||c||c|}
\hline \phantom{sp} n \phantom{sp} & \phantom{sp} p \phantom{sp} & \phantom{sp}q \phantom{sp}& \phantom{s} $\neg q$ \phantom{s}& \phantom{s} $p \vee \neg q$ \phantom{s}\\  \hline
\hline 1 & 0 & 0 & 1 & \\ 
\hline 2 & 1 & 0 & 1 & \\ 
\hline 3 & 0 & 0 & 1 & \\ 
\hline 4 & 1 & 0 & 1 & \\ 
\hline 5 & 0 & 1 & 0 & \\ 
\hline 6 & 1 & 1 & 0 & \\ 
\hline 7 & 0 & 1 & 0 & \\ 
\hline 8 & 1 & 1 & 0 & \\ 
\hline 9 & 0 & 1 & 0 & \\ 
\hline 
\end{tabular}
\end{center} 
%\[\mbox{Truth Set} = \{1,3,5,6,7,8,9\}\]

%--------------------------------------------------- %
%--------------------------------------------------- %
%\subsection*{1.5 Truth Sets}

{Truth Sets}

%% - \vspace{-0.5cm}
\begin{center}
\begin{tabular}{|c||c|c||c||c|}
\hline \phantom{sp} n \phantom{sp} & \phantom{sp} p \phantom{sp} & \phantom{sp}q \phantom{sp}& \phantom{s} $\neg q$ \phantom{s}& \phantom{s} $p \vee \neg q$ \phantom{s}\\  \hline
\hline 1 & 0 & 0 & 1 & 1\\ 
\hline 2 & 1 & 0 & 1 & 0\\ 
\hline 3 & 0 & 0 & 1 & 1\\ 
\hline 4 & 1 & 0 & 1 & 0\\ 
\hline 5 & 0 & 1 & 0 & 1\\ 
\hline 6 & 1 & 1 & 0 & 1\\ 
\hline 7 & 0 & 1 & 0 & 1\\ 
\hline 8 & 1 & 1 & 0 & 1\\ 
\hline 9 & 0 & 1 & 0 & 1\\ 
\hline 
\end{tabular}
\end{center} 
\[\mbox{Truth Set} = \{1,3,5,6,7,8,9\}\]


%--------------------------------------------------- %
%--------------------------------------------------- 

\subsection*{1.5 Truth Sets}

{Truth Sets}

\begin{center}
\begin{tabular}{|c||c|c||c||c|}
\hline \phantom{sp} n \phantom{sp} & \phantom{sp} p \phantom{sp} & \phantom{sp}q \phantom{sp}& \phantom{s} $  p \rightarrow q$ \phantom{s}& \phantom{s} $q \rightarrow p$ \phantom{s}\\  \hline
\hline 1 & 0 & 0 & 1 & 0\\ 
\hline 2 & 1 & 0 & 1 & 0\\ 
\hline 3 & 0 & 0 & 1 & 0\\ 
\hline 4 & 1 & 0 & 1 & 0\\ 
\hline 5 & 0 & 1 & 0 & 1\\ 
\hline 6 & 1 & 1 & 1 & 0\\ 
\hline 7 & 0 & 1 & 0 & 1\\ 
\hline 8 & 1 & 1 & 1 & 0\\ 
\hline 9 & 0 & 1 & 0 & 1\\ 
\hline 
\end{tabular} 
\end{center}
\[\mbox{Truth Set} = \{5,7,9\}\]

%--------------------------------------------------- %
%--------------------------------------------------- %
%\subsection*{1.5 Truth Sets}





%--------------table
%\begin{tabular}{cccc}\hline
%p & q& P \vee q& p \wedge q  \\
%0& 0& 0& 0\\ 
%0& 1& 0& 1 \\ 
%1& 0& 0& 1 \\ 
%1& 1&1 &1  \\  \hline
%\end{tabular}
\newpage

%--------------------------------------------%
\subsection*{Logic Networks}
\begin{itemize}
\item AND Gates
\item OR Gates
\item NOT Gates
\end{itemize}

\newpage

%--------------------------------------------------- %
\subsection*{1.2 2010 Question 3}

Let p and q be propositions. Use Truth Tables to prove that

\[ p \rightarrow q \equiv \neg q \rightarrow \neg\]
\textbf{Important} Remember to make a comment at the end to say why the table proves that the two statements are logically equivalent. e.g. \emph{‘since the columns are identical both sides of the equation are equivalent’}.
{ 
\begin{tabular}{|c|c||c|}
\hline  p&  q& $p \rightarrow q$ \\ 
\hline  0&  0&  1\\ 
\hline  0&  1&  1\\ 
\hline  1&  0&  0\\ 
\hline  1&  1&  1\\ 
\hline 
\end{tabular} \hspace{0.5cm} \begin{tabular}{|c|c||c|c|c|}
\hline  p&  q& $\neg q$ & $\neg p$ & $\neg q \rightarrow \neg p$ \\ 
\hline  0&  0& 1& 1& 1\\ 
\hline  0&  1& 0& 1& 1\\ 
\hline  1&  0& 1& 0& 0\\ 
\hline  1&  1& 0& 0& 1\\ 
\hline 
\end{tabular}
} 
(Key ``difference" is first and last rows)
%--------------------------------------------------- %


\end{document}
