
\documentclass[12pt]{article}
%\usepackage[final]{pdfpages}

\usepackage{graphicx}
\graphicspath{{/Users/kevinhayes/Documents/teaching/images/}}

\usepackage{tikz}
\usetikzlibrary{arrows}

\newcommand{\bbr}{\Bbb{R}}
\newcommand{\zn}{\Bbb{Z}^n}

%\usepackage{epsfig}
%\usepackage{subfigure}
\usepackage{amscd}
\usepackage{amssymb}
\usepackage{amsbsy}
\usepackage{amsthm}
\usepackage{natbib}
\usepackage{amsbsy}
\usepackage{enumerate}
\usepackage{amsmath}
\usepackage{eurosym}
%\usepackage{beamerarticle}
\usepackage{txfonts}
\usepackage{fancyvrb}
\usepackage{fancyhdr}
\usepackage{natbib}
\bibliographystyle{chicago}

\usepackage{vmargin}
% left top textwidth textheight headheight
% headsep footheight footskip
\setmargins{2.0cm}{2.5cm}{16 cm}{22cm}{0.5cm}{0cm}{1cm}{1cm}
\renewcommand{\baselinestretch}{1.3}


\pagenumbering{arabic}

\begin{document}

%--------------------------------------------------- %
%--------------------------------------------------- %
\subsection*{1.7 2008 Q3b Logic Networks }

Construct a logic network that accepts as input p and q, which may independently have the value 0 or 1, and
gives as final input $\neg(p \wedge \not q)$ (i.e. $\equiv p \rightarrow q$).\\
\bigskip

\textbf{Logic Gates}
\begin{itemize}
\item AND
\item OR
\item NOT
\end{itemize}
\bigskip

\emph{\textbf{Examiner's Comments:}Many
diagrams were carefully and clearly drawn and well labelled, gaining full
marks. The logic table was also well done by most, but there were a few marks
lost in the final part by failing to deduce that ‘since the columns of the table are
identical the expressions are equivalent’.}

%--------------------------------------------------- %
%--------------------------------------------------- %
\subsection*{1.8 2008 Q3b Logic Networks }
Construct a logic network that accepts as input p and q, which may independently have the value 0 or 1, and
gives as final input $(p \wedge  q) \vee \neg q$ (i.e. $\equiv p \rightarrow q$).



\textbf{Important} Label each of the gates appropriately and label the diagram with a symblic expression for the output after each gate.




\textbf{Logic Gates}
\begin{itemize}
\item AND
\item OR
\item NOT
\end{itemize}
\bigskip

\emph{\textbf{Examiner's Comments:}Many
diagrams were carefully and clearly drawn and well labelled, gaining full
marks. The logic table was also well done by most, but there were a few marks
lost in the final part by failing to deduce that ‘since the columns of the table are
identical the expressions are equivalent’.}

%--------------------------------------------------- %
%--------------------------------------------------- %
\subsection*{1.8 2008 Q3b Logic Networks }
Construct a logic network that accepts as input p and q, which may independently have the value 0 or 1, and
gives as final input $(p \wedge  q) \vee \neg q$ (i.e. $\equiv p \rightarrow q$).



\textbf{Important} Label each of the gates appropriately and label the diagram with a symblic expression for the output after each gate.





\end{document}
