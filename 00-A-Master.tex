
%--------------------------------------------------- %
\subsection*{1.3 Membership Tables for Laws}
\emph{Page 44 (Volume 1) Q8.
Also see Section 3.3 Laws of Logic.}\\

Construct a truth table for each of the following compound statement and hence find simpler propositions to which it is equivalent.


\begin{itemize}
\item $p \vee F$
\item $p \wedge T$
\end{itemize}
\textbf{Solutions}
\begin{center}
{
\begin{tabular}{|c|c||c|c|}
\hline  p & T & $p \vee T$ & $ p \wedge T$ \\ \hline
\hline  0 & 1 & 1 & 0 \\ 
\hline  1 &  1 & 1 & 1 \\ 
\hline 
\end{tabular} 
}
\end{center}
\begin{itemize}
\item Logical OR: $p \vee T = T $
\item Logical AND: $p \wedge T = p  $
\end{itemize}

%-------------------------------------%
\begin{center}
{
\begin{tabular}{|c|c||c|c|}
\hline  p & F & $p \vee F$ & $ p \wedge F$ \\ \hline
\hline  0 & 0 & 0 & 0 \\ 
\hline  1 &  0 & 1 & 0 \\ 
\hline 
\end{tabular} 
}
\end{center}
\begin{itemize}
\item Logical OR:  $p \vee F = p $
\item Logical AND: $p \wedge F = F $
\end{itemize}
%--------------------------------------------------- %

\section*{De Morgan's Laws}
The De Morgan's Laws allow the expression of conjunctions and disjunctions purely in terms of each other via negation.
\bigskip

\noindent For two propositions A and B, the laws can be verbalized as:
\begin{itemize}
\item The negation of a conjunction is the disjunction of the negations.
\item The negation of a disjunction is the conjunction of the negations.
\end{itemize}

\begin{framed}
\noindent Using Pseudo-Notation
\begin{itemize}
	\item[(i)]"not (A and B)" is the same as "(not A) or (not B)"
	
	\item[(ii)] "\textbf{not (A or B)}" is the same as "\textbf{(not A) and (not B)}"
\end{itemize}
\end{framed}
%------------------------------------------------------------ %
\newpage

\subsection{Exercise }Use Truth Tables to prove De Morgan's Laws.
{
	\LARGE
\[  \neg (p \vee q) = \neg p \wedge \neg q\]

}
{
	
\Large
\begin{center}
\begin{tabular}{|c|c||c|c|c|c|}
  \hline
  % after \\: \hline or \cline{col1-col2} \cline{col3-col4} ...
p	&	q	&	$ p \vee q$	&	$ p \wedge q$&	$\neg (p \vee q)$	&	$\neg (p \wedge q)$\\
	&		&	(1)	&	(2)	&	(3)	&	(4)	\\ \hline
\phantom{sp}0\phantom{sp}	&	\phantom{sp}0\phantom{sp}	&	\phantom{sp}0\phantom{sp}	&	\phantom{sp}0\phantom{sp}	&	1	&	1 \\
0	&	1	&	1	&	0	&	0	&	1\\
1	&	0	&	1	&	0	&	0	&	1\\
1	&	1	&	1	&	1	&	\phantom{sp}0\phantom{sp}	&	\phantom{sp}0\phantom{sp}\\
  \hline
\end{tabular}
\end{center}
}
\newpage

\noindent	Looking at the lefthand side of equation
{\LARGE	\[  \neg (p \vee q) = \neg p \wedge \neg q\]
}	
\bigskip
{
	\Large
\begin{center}
	\begin{tabular}{|c|c||c|c|c|c|}
		\hline
		% after \\: \hline or \cline{col1-col2} \cline{col3-col4} ...
		p	&	q	&	$ p \vee q$	&	$ p \wedge q$&	$\neg (p \vee q)$	&	$\neg (p \wedge q)$\\
		&		&	(1)	&	(2)	&	(3)	&	(4)	\\ \hline
		\phantom{sp}0\phantom{sp}	&	\phantom{sp}0\phantom{sp}	&	\phantom{sp}0\phantom{sp}	&	\phantom{sp}0\phantom{sp}	&	1	&	1 \\
		0	&	1	&	1	&	0	&	0	&	1\\
		1	&	0	&	1	&	0	&	0	&	1\\
		1	&	1	&	1	&	1	&	\phantom{sp}0\phantom{sp}	&	\phantom{sp}0\phantom{sp}\\
		\hline
	\end{tabular}
\end{center}
}
\bigskip
\noindent	Looking at the righthand side of equation
	\[  \neg (p \vee q) = \neg p \wedge \neg q\]
	\begin{center}
		\begin{tabular}{|c|c||c|c|c|c|}
			\hline
			% after \\: \hline or \cline{col1-col2} \cline{col3-col4} ...
			p	&	q	&	$\neg$p	&	$\neg$q	&	$\neg p \wedge \neg q$	&	$\neg p \vee \neg q$ \\ 
			&		&	(5)	&	(6)	&	(7)	&	(8)	\\
			\hline
			\phantom{sp}0\phantom{sp}	&	\phantom{sp}0\phantom{sp}	&	1	&	1	&	1	&	1	\\
			0	&	1	&	1	&	0	&	0	&	1	\\
			1	&	0	&	0	&	1	&	0	&	1	\\
			1	&	1	&	\phantom{sp}0\phantom{sp}	&	\phantom{sp}0\phantom{sp}	&	\phantom{sp}0\phantom{sp}	&	\phantom{sp}0\phantom{sp}	\\
			\hline
		\end{tabular}
	\end{center}


%============================================================= %
