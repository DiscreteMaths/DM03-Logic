\documentclass[12pt]{report}

\usepackage{amsmath}
\usepackage{amssymb}
\usepackage{graphics}
\usepackage{framed}


\begin{document}

\section*{Laws of Logic}

\subsection*{De Morgan's Laws}

In propositional logic and boolean algebra, \textbf{De Morgan's laws} are a pair of transformation rules that are both valid rules of inference. The rules allow the expression of conjunctions and disjunctions purely in terms of each other via negation.

\noindent The rules can be verbalized as:

\begin{itemize}
\item The negation of a conjunction is the disjunction of the negations.
\item The negation of a disjunction is the conjunction of the negations.
\end{itemize}
or informally as:
\begin{center}
\textbf{``not (A and B)" is the same as "(not A) or (not B)"
}
\end{center}

and also,
\begin{center}
\textbf{``not (A or B)" is the same as "(not A) and (not B)"
}
\end{center}
The rules can be expressed in formal language with two propositions P and Q as:
\[ \neg(P\land Q)\iff(\neg P)\lor(\neg Q)\]
\[\neg(P\lor Q)\iff(\neg P)\land(\neg Q)\]
where:
\begin{itemize}
\item $\neg$ is the negation operator (NOT)
\item $\wedge $ is the conjunction operator (AND)
\item $\vee$ is the disjunction operator (OR)
\item $\leftarrow \rightarrow$ is a metalogical symbol meaning "can be replaced in a logical proof with"
\end{itemize}

\subsubsection{Set theory and Boolean algebra}
In set theory and Boolean algebra, De Morgan's Laws are often stated as "Union and intersection interchange under complementation" which can be formally expressed as:

\[\overline{A \cup B}\equiv\overline{A} \cap \overline{B}\]
\[\overline{A \cap B}\equiv\overline{A} \cup \overline{B}\]
where:
\begin{itemize}
\item $\overline{A}$ is the negation of A, the overline being written above the terms to be negated
\item $\cap$ is the intersection operator (AND)
\item $\cup$ is the union operator (OR)
\end{itemize}

\newpage



\section*{Laws of Logic}

\subsection*{De Morgan's Laws}

In propositional logic and boolean algebra, \textbf{De Morgan's laws} are a pair of transformation rules that are both valid rules of inference. The rules allow the expression of conjunctions and disjunctions purely in terms of each other via negation.

\noindent The rules can be verbalized as:

\begin{itemize}
\item The negation of a conjunction is the disjunction of the negations.
\item The negation of a disjunction is the conjunction of the negations.
\end{itemize}
or informally as:
\begin{center}
\textbf{``not (A and B)" is the same as "(not A) or (not B)"
}
\end{center}

and also,
\begin{center}
\textbf{``not (A or B)" is the same as "(not A) and (not B)"
}
\end{center}
The rules can be expressed in formal language with two propositions P and Q as:
\[ \neg(P\land Q)\iff(\neg P)\lor(\neg Q)\]
\[\neg(P\lor Q)\iff(\neg P)\land(\neg Q)\]
where:
\begin{itemize}
\item $\neg$ is the negation operator (NOT)
\item $\wedge $ is the conjunction operator (AND)
\item $\vee$ is the disjunction operator (OR)
\item $\leftarrow \rightarrow$ is a metalogical symbol meaning "can be replaced in a logical proof with"
\end{itemize}

\subsubsection{Set theory and Boolean algebra}
In set theory and Boolean algebra, De Morgan's Laws are often stated as "Union and intersection interchange under complementation" which can be formally expressed as:

\[\overline{A \cup B}\equiv\overline{A} \cap \overline{B}\]
\[\overline{A \cap B}\equiv\overline{A} \cup \overline{B}\]
where:
\begin{itemize}
\item $\overline{A}$ is the negation of A, the overline being written above the terms to be negated
\item $\cap$ is the intersection operator (AND)
\item $\cup$ is the union operator (OR)
\end{itemize}
\end{document}
%--------------------------------------------------- %
\subsection*{1.3 Membership Tables for Laws}
\emph{Page 44 (Volume 1) Q8.
Also see Section 3.3 Laws of Logic.}\\

Construct a truth table for each of the following compound statement and hence find simpler propositions to which it is equivalent.


\begin{itemize}
\item $p \vee F$
\item $p \wedge T$
\end{itemize}
\textbf{Solutions}
\begin{center}
{
\begin{tabular}{|c|c||c|c|}
\hline  p & T & $p \vee T$ & $ p \wedge T$ \\ \hline
\hline  0 & 1 & 1 & 0 \\ 
\hline  1 &  1 & 1 & 1 \\ 
\hline 
\end{tabular} 
}
\end{center}
\begin{itemize}
\item Logical OR: $p \vee T = T $
\item Logical AND: $p \wedge T = p  $
\end{itemize}

%-------------------------------------%
\begin{center}
{
\begin{tabular}{|c|c||c|c|}
\hline  p & F & $p \vee F$ & $ p \wedge F$ \\ \hline
\hline  0 & 0 & 0 & 0 \\ 
\hline  1 &  0 & 1 & 0 \\ 
\hline 
\end{tabular} 
}
\end{center}
\begin{itemize}
\item Logical OR:  $p \vee F = p $
\item Logical AND: $p \wedge F = F $
\end{itemize}
%--------------------------------------------------- %

\section*{De Morgan's Laws}
The De Morgan's Laws allow the expression of conjunctions and disjunctions purely in terms of each other via negation.
\bigskip

\noindent For two propositions A and B, the laws can be verbalized as:
\begin{itemize}
\item The negation of a conjunction is the disjunction of the negations.
\item The negation of a disjunction is the conjunction of the negations.
\end{itemize}

\begin{framed}
\noindent Using Pseudo-Notation
\begin{itemize}
	\item[(i)]"not (A and B)" is the same as "(not A) or (not B)"
	
	\item[(ii)] "\textbf{not (A or B)}" is the same as "\textbf{(not A) and (not B)}"
\end{itemize}
\end{framed}
%------------------------------------------------------------ %
\newpage

\subsection{Exercise }Use Truth Tables to prove De Morgan's Laws.
{
	\LARGE
\[  \neg (p \vee q) = \neg p \wedge \neg q\]

}
{
	
\Large
\begin{center}
\begin{tabular}{|c|c||c|c|c|c|}
  \hline
  % after \\: \hline or \cline{col1-col2} \cline{col3-col4} ...
p	&	q	&	$ p \vee q$	&	$ p \wedge q$&	$\neg (p \vee q)$	&	$\neg (p \wedge q)$\\
	&		&	(1)	&	(2)	&	(3)	&	(4)	\\ \hline
\phantom{sp}0\phantom{sp}	&	\phantom{sp}0\phantom{sp}	&	\phantom{sp}0\phantom{sp}	&	\phantom{sp}0\phantom{sp}	&	1	&	1 \\
0	&	1	&	1	&	0	&	0	&	1\\
1	&	0	&	1	&	0	&	0	&	1\\
1	&	1	&	1	&	1	&	\phantom{sp}0\phantom{sp}	&	\phantom{sp}0\phantom{sp}\\
  \hline
\end{tabular}
\end{center}
}
\newpage

\noindent	Looking at the lefthand side of equation
{\LARGE	\[  \neg (p \vee q) = \neg p \wedge \neg q\]
}	
\bigskip
{
	\Large
\begin{center}
	\begin{tabular}{|c|c||c|c|c|c|}
		\hline
		% after \\: \hline or \cline{col1-col2} \cline{col3-col4} ...
		p	&	q	&	$ p \vee q$	&	$ p \wedge q$&	$\neg (p \vee q)$	&	$\neg (p \wedge q)$\\
		&		&	(1)	&	(2)	&	(3)	&	(4)	\\ \hline
		\phantom{sp}0\phantom{sp}	&	\phantom{sp}0\phantom{sp}	&	\phantom{sp}0\phantom{sp}	&	\phantom{sp}0\phantom{sp}	&	1	&	1 \\
		0	&	1	&	1	&	0	&	0	&	1\\
		1	&	0	&	1	&	0	&	0	&	1\\
		1	&	1	&	1	&	1	&	\phantom{sp}0\phantom{sp}	&	\phantom{sp}0\phantom{sp}\\
		\hline
	\end{tabular}
\end{center}
}
\bigskip
\noindent	Looking at the righthand side of equation
	\[  \neg (p \vee q) = \neg p \wedge \neg q\]
	\begin{center}
		\begin{tabular}{|c|c||c|c|c|c|}
			\hline
			% after \\: \hline or \cline{col1-col2} \cline{col3-col4} ...
			p	&	q	&	$\neg$p	&	$\neg$q	&	$\neg p \wedge \neg q$	&	$\neg p \vee \neg q$ \\ 
			&		&	(5)	&	(6)	&	(7)	&	(8)	\\
			\hline
			\phantom{sp}0\phantom{sp}	&	\phantom{sp}0\phantom{sp}	&	1	&	1	&	1	&	1	\\
			0	&	1	&	1	&	0	&	0	&	1	\\
			1	&	0	&	0	&	1	&	0	&	1	\\
			1	&	1	&	\phantom{sp}0\phantom{sp}	&	\phantom{sp}0\phantom{sp}	&	\phantom{sp}0\phantom{sp}	&	\phantom{sp}0\phantom{sp}	\\
			\hline
		\end{tabular}
	\end{center}


%============================================================= %
