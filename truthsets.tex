

%- Tautologies and Contradiction
%- Double Negative Law
%- De Morgan's LAw
%- Compound Statements
%- Truth Tables
%- Conditional Connectives
%--------------------------------------------------------%

%--------------------------------------%

\section{Logic Proposition}
Let $p$, $Q$ and $r$ be the following propositions concerning integers $n$:

\begin{itemize}
\item $p$ : n is a factor of 36 (2)
\item $q$ : n is a factor of 4 (2)
\item $r$ : n is a factor of 9 (3)
\end{itemize}


\begin{center}
\begin{tabular}{|c||c|c|c|}
\hline 
\phantom{spa} \textbf{n} \phantom{spa}	& \phantom{spa}	\textbf{p} \phantom{spa}	& \phantom{spa}	\textbf{q} \phantom{spa}	& \phantom{spa}	\textbf{r} \phantom{spa}	\\ \hline \hline
1	&	1	&	1	&	1	\\ \hline
2	&	1	&	0	&	1	\\ \hline
3	&	0	&	1	&	1	\\ \hline
4	&	1	&	0	&	1	\\ \hline
6	&	0	&	0	&	1	\\ \hline
9	&	0	&	1	&	1	\\ \hline
12	&	0	&	0	&	1	\\ \hline
18	&	0	&	0	&	1	\\ \hline
36	&	0	&	0	&	1	\\
\hline 
\end{tabular} 

For each of the following compound statements, express it using the propositions P q and r, andng logical symbols, then given the truth table for it,

\begin{itemize}
\item[1)] If n is a factor of 36, then n is a factor of 4 or n is a factor of 9
\item[2)] If n is a factor of 4 or n is a factor of 9 then  n is a factor of 36
\end{itemize}

%--------------------------------------%

\newpage

\section*{Part 1 : Logic}


\subsection*{1.1 2010 Question 3}
Let $S = \{10,11,12,13,14,15,16,17,18,19\}$ and let p, q be the following propositions concerning the integer $n \in S$.

\begin{itemize}
\item p: n is a multiple of two. (i.,18e. $\{10,12,14,16,18\}$)
\item q: n is a multiple of three. {i.e. $\{12,15,18\}$}
\end{itemize}

For each of the following compound statements find the sets of values n for which it is true. 

\begin{itemize}
\item $p \vee q$ : (p or q :  10 12 14 15 16 18) 
\item $p \wedge q$: (p and q: 12 18)
\item $ \neg p \oplus q$: (not-p or q, but not both)
\begin{itemize}
\item $\neg p $ not-p = $\{ 11 13 15 17 19\}$
\item $\neg p \vee q$ not-p or q $\{11 12 13 15 17 18 19\}$
\item $\neg p \wedge q$ not-p and q $\{15\} $
\item $ \neg p \oplus q$ = $\{11, 12, 13, 17, 18, 19\}$
\end{itemize}
\end{itemize}

%--------------------------------------------------- %
%--------------------------------------------------- %
\subsection*{1.5 Truth Sets}
\textbf{2009} 

Let $n = \{1, 2,3,4, 5,6,7, 8, 9\}$ and let p, q be the following propositions concerning the integer $n$.
\begin{itemize}
\item p: n is even, 
\item q: $n\geq 5$.
\end{itemize}
By drawing up the appropriate truth table find the truth set for each of the
propositions $p \vee \neg q$ and $ \neg q \rightarrow p$

\begin{tabular}{|c|c|c|c|c|}
\hline n & p & q & $\neg q$ & $p \vee \neg q$ \\ 
\hline 1 & 0 & 0 & 1 & 1\\ 
\hline 2 & 1 & 0 & 1 & 0\\ 
\hline 3 & 0 & 0 & 1 & 1\\ 
\hline 4 & 1 & 0 & 1 & 0\\ 
\hline 5 & 0 & 1 & 0 & 1\\ 
\hline 6 & 1 & 1 & 0 & 1\\ 
\hline 7 & 0 & 1 & 0 & 1\\ 
\hline 8 & 1 & 1 & 0 & 1\\ 
\hline 9 & 0 & 1 & 0 & 1\\ 
\hline 
\end{tabular} 
\[\mbox{Truth Set} = \{1,3,5,6,7,8,9\}\]

\begin{tabular}{|c|c|c|c|c|}
\hline n & p & q & $  q \rightarrow p$ & $  q \rightarrow p$ \\ 
\hline 1 & 0 & 0 & 1 & 0\\ 
\hline 2 & 1 & 0 & 1 & 0\\ 
\hline 3 & 0 & 0 & 1 & 0\\ 
\hline 4 & 1 & 0 & 1 & 0\\ 
\hline 5 & 0 & 1 & 0 & 1\\ 
\hline 6 & 1 & 1 & 1 & 0\\ 
\hline 7 & 0 & 1 & 0 & 1\\ 
\hline 8 & 1 & 1 & 1 & 0\\ 
\hline 9 & 0 & 1 & 0 & 1\\ 
\hline 
\end{tabular} 
\[\mbox{Truth Set} = \{5,7,9\}\]
%---------------------------------------
%--------------------------------------------------- %
%--------------------------------------------------- %
